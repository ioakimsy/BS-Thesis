% !TEX root =  main.tex
\chapter{Peer Instruction and Traditional Models of Teaching}

Most of the literature on peer instruction (PI) for physics education references Eric Mazur's work \textit{Peer Instruction: A User's Manual} \cite{mazur1997peer,mazur1999}.
Since he started teaching physics in Harvard University in 1984, he has found that although students can memorize the laws and equations of physics and apply them in numerical problem-solving, there is a lack of understanding of the concepts behind the equations.
Although his students can solve traditional quantitative problems and score high, they scored lower when given conceptual qualitative questions.
Performing some statistical analysis on the scores of his students, he found that although 52\% of his students did similarly well on conceptual and conventional problems, 39\% of them did substantially worse on the conceptual problems than the conventional problems, while only 9\% did substantially better on the conceptual problems.
From this, he concluded the way students approached these physics courses was to memorize problem-solving algorithms that don't even work with all problems.
He said that this explains why his bright students sometimes blundered and why students generally get frustrated with physics.

The rationale behind PI is that students have to think for themselves during the discussion part of the class where they have to convince their seatmates that their answer is correct.
In this way, students don't simply absorb the information given to them as they would during traditional lectures.
When students engage learning materials actively and cooperatively, they develop good complex reasoning skills effectively - this applies to higher education students as well \cite{johnson2008active}.
Despite having its roots in introductory physics courses, PI has also seen success in higher-level courses \cite{fagen2000factors, fagen2002peer}.


\section{Difference of Peer Instruction from Traditional Models}\label{sec:PI vs TI}
PI has been described as a student-centered approach to teaching. 
It aims to make the students actively engage with the material they are given more so than traditional lectures.
Although there are many ways to implement PI, most methods involve shortening lectures delivered by the instructor, giving conceptual tests to gauge student understanding, and then allowing the students to interact and learn from each other \cite{crouch2001peer,mazur1999}.

Considering that the goal of PI is to have more and deeper student engagement, the way it is implemented varies from traditional lectures.
Most implementations of PI revolve around five things \cite{crouch2001peer,mazur1999}.

First (1), students are expected to read the material before class.
This pre-class learning opportunity allows the instructor to spend less time on delivering presentations to teach the material.
It would also be beneficial if the materials that were given to students are written to be read before class.
This is unfortunately not the case for most textbooks which are written mostly as something to be reflectively read after a lecture.

Second (2), short presentations that usually given for each central point of the lesson.
They are meant to contain little derivations, focus on the strategy on how to solve a problem from start to finish while highlighting the conceptual significance of each step.
Because students are expected to have read up on the lesson, the instructor can also spend less time on definitions, especially those that are written in the book.
In the case of physics education, example problems should ideally be quantitative to provide maximum physical insight while minimizing algebra.

Third (3) are qualitative and multiple choice concept tests. 
It is important for concept tests to be designed properly because they are what helps the instructor gauge which parts of the lesson are hard for students to understand.
They also serve as guides for what the students can discuss among themselves.
In the paper written by Crouch, et al. \cite{crouch2001peer}, which had a modified version of Mazur's original peer instruction method \cite{mazur1999}, these concept tests are not graded, but students are given incentives when they participate regularly.
Aside from incentives, the concept tests were also made to be a part of the midterm and final exams, which gave more reasons for students to participate in them.

Fourth (4), peer discussion.
After the concept test, students discuss their answers with those around them.
They are encouraged to explain to others the correctness of their own answers by explaining the reasoning behind their answer.
During this time, the instructor can walk around to listen in on student discussions to gain insights on students' understanding of the topic.

Fifth (5), at the end of peer discussion, another concept test on the same central point is given.
The initial and final concept tests, as well as insights from the peer discussions should guide the instructor on what to discuss and how much time to spend on the discussion.
This discussion should focus mainly on explaining the correct answer to the concept test.

These parts or steps are not meant to be rigid. Since PI is supposed to be a student-centric approach, instructors should give considerations and adjust accordingly. 
The time spent on the concept tests and discussions should be varied based on the class as well as the difficulty of the topic. 
One can also vary the flow of discussion based on the percentage of correct answers. 
Lasry et al. \cite{lasry2008peer} gave an example on how this can be done.
If only a few students get the correct answer, the instructor can revisit the concept before giving another concept test.
If most of the students get the right answer, the instructor can simply explain the correct answer before moving on to the next central point.
In the case that the percentage of correct answers lie somewhere in the middle, the instructor can hold a peer discussion session before asking the students to revote on the concept test, which is then followed by a quick explanation from the instructor before moving on.

% Bullet point notes
% \begin{itemize}
%     \item The goal of peer instruction is to have students engage with other students in their class and discuss the core concepts that were discussed in class \cite{crouch2001peer}.
%     \item Contrary to traditional lectures, PI makes it so that all students are forced to engage, unlike in traditional lectures where only highly motivated students usually engage \cite{crouch2001peer}.
%     \item Short presentation → concept test -> peer discussion -> concept test -> exam items. ! not graded ! \cite{crouch2001peer}
%     \item Pre-class readings make it so that the focus of the lectures are on the difficult parts \cite{crouch2001peer}. Pre-readings that were used were different from ordinary textbooks where they were meant to be read reflectively after class.
%     \item 1/3 to 1/2 of the class is devoted to concept tests and the rest on lectures. Time varies based on the class and difficulty of topic.
%     \item Instructors can choose to cover less but deeper, or rely on students to learn on their own when not all topics can be discussed in class.
%     \item Lectures contain little derivation. Focus on strategy on start to end of problems, highlighting the conceptual significance of each step.
%     \item Less time on definitions.
%     \item Quantitative examples for max physical insight and minimal algebra.
% \end{itemize}

\section{Benefits of Peer Instruction}
Despite the de-emphasis of problem-solving in PI lectures, the students' learning in quantitative problem-solving was not compromised and was even improved compared to traditional lectures \cite{crouch2001peer}.
In this case, the students' learning comes primarily from discussions with their peers and homework assignments.
They found that using PI significantly increased the students' scores on the quantitative problems.
PI also decreased the number of students who scored extremely low on the quantitative problems.
Their findings are consistent with the findings of other studies \cite{thacker1994comparing,lasry2008peer}.
Similarly, there was a significant increase in the students' conceptual understanding of the topics discussed in class \cite{crouch2001peer}.
This improvement holds true for both in-class concept tests and also for end-of-semester exams.
Lasry et al. \cite{lasry2008peer} affirms this finding by showing that PI provides better conceptual understanding to students while providing the same level of quantitative problem-solving skills as traditional instruction.

One concern for PI is that the student may not be learning from peer interactions, but simply choosing the same answer as those who they think are more knowledgeable.
This hypothesis was tested by Smith et al. \cite{smith2009peer} by using isomorphic questions.
Isomorphic questions are questions that have different ``cover stories" but have the same underlying concept.
By having a pair of isomorphic questions for the test-share-test process and by not showing the class's answer statistics, they found that the students actually learn from the peer interactions and don't just copy others' answers.

Furthermore, they found that for groups of students who did not get the initial question right (naïve), a significant number got the second question right despite not knowing the answer to the first question.
This is in line with students' opinions wherein they say that a student who knows the answer to the initial question was not required for the discussions to be productive.
In fact, some even shared that they learned better when there is no one in the group that knew the answer because in this way, they are forced to work through and discuss the problem.
This finding contradicts the view that the value of PI comes from being ``transmissionist" where students only learn from learned individuals, either students or instructors.
The study done by Smith et al. \cite{smith2009peer} provides evidence that PI is more constructivist rather than transmissionist - meaning that the students are learning on their own through discussion and not from simply hearing the correct answer.

Besides those mentioned above, Lasry et al. \cite{lasry2008peer} tackles two other things: the dependence of PI on student background knowledge and the effect of PI on student attrition.
They found that PI, regardless of students' background knowledge, performs better or just as well as traditional instruction with students who have more background knowledge.
In a two-year college, they found that PI is more effective than traditional instruction for both high and low background knowledge students.
In a four-year college like Harvard University, they found that there was no correlation between pre-test scores and the gain in the post-test scores, unlike in the two-year college.
This difference in dependence tells us that the gains of PI is not universal and is hypothesized to be dependent on students' high reasoning abilities, as suggested by previous studies \cite{coletta2005interpreting}.
In the same study, it was found that PI significantly reduced the number of students who dropped out of the course.
PI helps reduce student attrition by shifting the focus and method of instruction of the courses.
A paper by Tobias \cite{tobias1990they} suggests that the competitive nature and the focus on skill performance are possible reasons for students dropping out in science courses.
Both these are combatted by PI by shifting the focus to conceptual understanding and cooperative learning.
As a consequence of the reduction of students dropping out, the failure rate of students also dropped significantly.



% \begin{itemize}
%     \item Increase both conceptual reasoning and problem-solving skills \cite{crouch2001peer}
%     \item Increase understanding even if no one initially knew the answer \cite{smith2009peer} 
%     \cite{lasry2008peer} (affirmation of previous findings)
%     \item More background knowledge benefit from PI and TI similarly. PI students with less background knowledge gain as much as students with more background knowledge in TI. PI more knowledge = TI more knowledge = PI less knowledge \cite{lasry2008peer}
%     \item PI applicable to top tier universities (Harvard) but also community/two-year colleges \cite{lasry2008peer}
%     \item Decrease student attrition (significantly less dropping)
% \end{itemize}

\section{Drawbacks of Peer Instruction}
As mentioned in Section~\ref{sec:PI vs TI}, prior to coming to the classroom, students are expected to have read up on the topic to help PI methods proceed smoother \cite{mazur1999,crouch2001peer}.
This can prove to be a disadvantage when instructors will have to rely on students' own discipline to do the tasks.
Although there are some things that can be done to help students, like providing guide questions or learning objective, instructors will still have to ultimately trust their students to do the work outside the classroom. 

Another draw back of PI that was outlined by Crouch et al. \cite{crouch2001peer} is that concept tests take up a lot of time.
Because of the amount of time it takes up, instructors will have to adjust by either (1) removing some topics from the course or (2) not having class sessions for some topics, leaving it to students to learn on their own via reading and problem sets.
A similar problem arises when we consider the classes' focus on conceptual understanding (which is part of classroom time management for PI.)
It will also be up to the students to study the things that were not tackled in-depth in class such as derivation and practice problems.

In addition to dependence on students' self-discipline and a different classroom time management style, the effectiveness of PI can also depend on students' background knowledge and reasoning capabilities \cite{lasry2008peer}.
Just as much as this dependence can be an advantage, it can also be a disadvantage.
Although previous studies show that PI does not really perform worse than traditional instruction \cite{crouch2001peer,lasry2008peer}, we should consider the time and resources that needs to be spent to adjust from the latter to the former and whether that is worth the non-uniform improvement.

% \begin{itemize}
%     \item Students are required to complete readings on the topics before class \cite{crouch2001peer}
%     \item Possible compromises on how many topics can be covered in class or relying on students to learn on their own (pre-class readings, derivations, etc.)
%     \item Can be dependent on student background knowledge. Higher knowledge = higher gain, PI and TI. This is not universal. Applied to community college but not Harvard \cite{lasry2008peer}
% \end{itemize}

\section{Existing mathematical models of PI}
Although there are some mathematical models for learning, the ones that describe learning in the classroom are few and far in between - even more so for those that model PI.

Roxas et al. \cite{roxas2010seating} used actual assessment results to train a neural network to map student interactions in PI classrooms. 
Using this neural network, they were able to characterize information transfer and investigate the effects of group homogeneity. 
Their study also investigated the optimal seating arrangement for students under PI methods based on their aptitude level.
In their paper, the measure of students' improvement was calculated via the Hake gain as shown in Equation~\ref{eq: hake gain} \cite{hake1998}.
They also used the output/input ratio (O/I), which was the ratio of second assessment scores vs first assessment scores, to gauge student improvement.
However, it should be noted that O/I values tend to be biased towards low-scoring students.
\begin{equation}
    \label{eq: hake gain}
    \langle g \rangle = \frac{\langle 2^{\text{nd}}\text{ assessment} - 1^{\text{st}}\text{ assessment} \rangle}{\langle 1 - 1^{\text{st}}\text{ assessment} \rangle}
\end{equation}

The results of their study show that the outer corner seating arrangement (SA) performed the best, followed by inner corner, then random, then center (see Figure~\ref{fig:PI SAs} for SA visualizations.)
In simulated classrooms, each with 64 students and 10 classrooms in total, they found that homogenous classrooms with low aptitudes have significantly higher O/I values.
This means that low aptitude students benefit the most from being grouped together.

Nitta \cite{nitta2019mathematical} gives us a few existing models that model PI. 
One of the models that were presented is a generalized Ising Model by Bordogna and Albano \cite{bordogna2001theoretical,bordogna2003simulation} where they consider three sources of information for the student to learn from: teacher instruction, peer interaction, and bibliographic materials (books, lecture notes, etc.)
Their model shows that students learn more when they engage discussions with their peers than those who only listen to lectures.
They also show that group structure affects student learning, and that low aptitude students may learn at the expense of high aptitude peers - a transmissionist view of PI.

Nitta also presents a model by Pritchard et al \cite{pritchard2008mathematical} where PI is modeled as a set differential equation that is dependent on the probability of students learning to stick (memory model, Equation~\ref{eq:memory model}) and the ability for students to associate new learnings from old knowledge via logistic differential equation (connectedness model, Equation~\ref{eq:connectedness model}.)

\begin{subequations}
    Memory model:
    \begin{equation}
        \label{eq:memory model}
        \frac{dU_T(t)}{dt} = -\alpha_m U_T(t)
    \end{equation}

    Connectedness model:
    \begin{equation}
        \label{eq:connectedness model}
        \frac{dU_T(t)}{dt} = -\alpha_c U_T(t)K_T(t)
    \end{equation}

    where knowledge is taken to grow at a uniform rate, as in the tutoring model:
    \begin{equation}
        \label{eq:tutoring model}
        K_T(t) = a_{tu}t + K_T(0)
    \end{equation}
    \begin{equation}
        U(t) + K(T) = 1
    \end{equation}
\end{subequations}

In these equations $U(T)$ and $K(T)$ are the unknown and known knowledge domains respectively. $\alpha_m$, $\alpha_c$, and $\alpha_{tu}$ are the corresponding rates for the memory model, connectedness model, and tutoring model 

In deriving their own equations to model PI, Nitta arrived at equations simialr to Hake gain to evaluate the effectiveness of PI for a concept test question and Pritchard's connectedness model to model the classes' learning processes.
Comparing their equations to data, they concluded that these metrics and equations roughly agree with the data and could give us insights on the learning dynamics of the classroom.

% \begin{itemize}
%     \item BA Ising Model - Monte carlo simulation. Considers multiple sources of information: teacher, peers, materials. PI + Lec > Lec only. Group structure affects learning. low PAL students learn at the expense of high PAL students
%     \item Pritchard et al model - differential equations described by probability of lessons to stick (memory model). Evolved into connectedness model by adding learning by association via logistic DE.
%     \item Nitta model of PI - similar to Pritchard et al connectedness model but derived through transition rates from wrong -> correct answers in concept tests. Admittedly a rough estimation, but one that agrees with data
% \end{itemize}

\section{Problem statement}
The process of PI is complex with a lot of parts that interact with each other.
Existing models are either predictive, as in the case of the neural network modeling of Roxas et al. \cite{roxas2010seating} or lack the spatial aspect of the process as with the differential equations of Pritchard \cite{pritchard2008mathematical} and Nitta \cite{nitta2019mathematical}.
While Bordogna et al. \cite{bordogna2001theoretical,bordogna2003simulation} present to us a dynamical model in their generalized Ising model, it lacks some of the aspects of PI we'd like to consider like seating arrangements, students' learning rate, and heterogeneity.

We propose that a probabilistic cellular automata model can be used to study the spatiotemporal dynamics of both PI and traditional instruction when incorporating these different aspects into the model.
