% !TEX root =  main.tex
\chapter{Modified PCA models}

\section{Heterogeneous Learning Rates}
To introduce learning rate heterogeneity in the classroom, we revisit equation \ref{eq:BPCA PI learning probability}. 
\begin{equation*}
    P_{ij} = 1 - \prod_{\forall \delta i, \delta j}{\lbrack1-(\lambda_{ij})(\rho_{i+\delta i, j+\delta j})(s_{i+\delta i, j+\delta j})}\rbrack
    \tag{\ref{eq:BPCA PI learning probability} revisted}
\end{equation*}
We can adjust the parameter $\lambda_{ij}$ to introduce differences in each student's learning rate. 
We set a student's learning rate as $\lambda_{ij} = \lambda_0 \pm \delta\lambda$ where $\delta\lambda \in \lbrace 0.0,0.1, 0.2, 0.3, 0.4\rbrace$ and $\lambda_0 = 0.5$. 
Each student has an equal chance of having a learning rate that is either faster $(\lambda_0 + \delta\lambda)$ or slower $(\lambda_0 - \delta\lambda)$ than the average learning rate $\lambda_0$. 

$\lambda_{i,j} \in [0,1]$ is the learning rate of student $c_{i,j}$ with values $\lambda_{i,j} = \lbrace \lambda_0 \pm \delta \lambda \rbrace$ where $\lambda_0 = 0.5$ and $\delta \lambda = \lbrace 0.1, 0.2, 0.3, 0.4 \rbrace$, 

\subsection{Effects on classroom evolution}\label{subsec:BPCAIH effects on classroom evolution}
We notice values of $\rho_0$ is still the most important factor in determining $t_{max}$. 
However, irregularities in the shapes of the "wave of learning" become more pronounced for lower values of $\rho_0$ and high values of $\delta\lambda$ as shown in figure \ref{fig:2DBPCAIH sample class evolution low rho}.
These irregularities are also more pronounced at the start of the classroom.
We also notice that compared to the homogenous model, there is a more evident deviation from the power law at earlier times $t$.
The irregularities in both the shape of the "wave of learning" and the deviation from the power law are less pronounced for higher values of $\rho_0$, regardless of the value of $\delta\lambda$, as shown in figure \ref{fig:2DBPCAIH sample class evolution high rho} and \ref{fig:2DBPCAIH sample class evolution high rho high delta}.
For high values of $\rho_0$, high values of $\delta\lambda$ leads to the irregularity in the "wave of learning" to persist longer than those with lower values of $\delta\lambda$.

\begin{figure}[htbp!]
    \centering
    \subfigure[$t=48$]{\label{fig:2DBPCAIH class evolution low rho 48}\includegraphics[width=0.45\textwidth]{figures/2D-BPCAIH-analysis/class evolutions/low rho/2DBPCAIH-inner_corner-128-0.1-0.5-0.4-trial_3-48.png}}
    \subfigure[$t=69$]{\label{fig:2DBPCAIH class evolution low rho 69}\includegraphics[width=0.45\textwidth]{figures/2D-BPCAIH-analysis/class evolutions/low rho/2DBPCAIH-inner_corner-128-0.1-0.5-0.4-trial_3-69.png}}
    \subfigure[$t=128$]{\label{fig:2DBPCAIH class evolution low rho 128}\includegraphics[width=0.45\textwidth]{figures/2D-BPCAIH-analysis/class evolutions/low rho/2DBPCAIH-inner_corner-128-0.1-0.5-0.4-trial_3-128.png}}
    \subfigure[$t=200$]{\label{fig:2DBPCAIH class evolution low rho 200}\includegraphics[width=0.45\textwidth]{figures/2D-BPCAIH-analysis/class evolutions/low rho/2DBPCAIH-inner_corner-128-0.1-0.5-0.4-trial_3-200.png}}
    \label{fig:2DBPCAIH sample class evolution low rho}
    \caption{Sample classroom evolutions for the inner corner configuration with $L=128$ at different times $t$ for positional learning rate $\rho_0=0.1$, $\delta\lambda = 0.4$.}
 \end{figure}

 \begin{figure}[htbp!]
    \centering
    \subfigure[$t=5$]{\label{fig:2DBPCAIH class evolution high rho 5}\includegraphics[width=0.45\textwidth]{figures/2D-BPCAIH-analysis/class evolutions/high rho/2DBPCAIH-inner_corner-128-0.9-0.5-0.1-trial_3-5.png}}
    \subfigure[$t=16$]{\label{fig:2DBPCAIH class evolution high rho 16}\includegraphics[width=0.45\textwidth]{figures/2D-BPCAIH-analysis/class evolutions/high rho/2DBPCAIH-inner_corner-128-0.9-0.5-0.1-trial_3-16.png}}
    \subfigure[$t=30$]{\label{fig:2DBPCAIH class evolution high rho 30}\includegraphics[width=0.45\textwidth]{figures/2D-BPCAIH-analysis/class evolutions/high rho/2DBPCAIH-inner_corner-128-0.9-0.5-0.1-trial_3-30.png}}
    \subfigure[$t=36$]{\label{fig:2DBPCAIH class evolution high rho 36}\includegraphics[width=0.45\textwidth]{figures/2D-BPCAIH-analysis/class evolutions/high rho/2DBPCAIH-inner_corner-128-0.9-0.5-0.1-trial_3-36.png}}   
    \label{fig:2DBPCAIH sample class evolution high rho}
    \caption{Sample classroom evolutions for the inner corner configuration with $L=128$ at different times $t$ for positional learning rate $\rho_0=0.9$, $\delta\lambda = 0.1$.}
 \end{figure}

 \begin{figure}[htbp!]
    \centering
    \subfigure[$t=5$]{\label{fig:2DBPCAIH class evolution high rho high delta 5}\includegraphics[width=0.45\textwidth]{figures/2D-BPCAIH-analysis/class evolutions/high rho high delta/2DBPCAIH-inner_corner-128-0.9-0.5-0.4-trial_3-5.png}}
    \subfigure[$t=13$]{\label{fig:2DBPCAIH class evolution high rho high delta 13}\includegraphics[width=0.45\textwidth]{figures/2D-BPCAIH-analysis/class evolutions/high rho high delta/2DBPCAIH-inner_corner-128-0.9-0.5-0.4-trial_3-13.png}}
    \subfigure[$t=19$]{\label{fig:2DBPCAIH class evolution high rho high delta 19}\includegraphics[width=0.45\textwidth]{figures/2D-BPCAIH-analysis/class evolutions/high rho high delta/2DBPCAIH-inner_corner-128-0.9-0.5-0.4-trial_3-19.png}}
    \subfigure[$t=29$]{\label{fig:2DBPCAIH class evolution high rho high delta 29}\includegraphics[width=0.45\textwidth]{figures/2D-BPCAIH-analysis/class evolutions/high rho high delta/2DBPCAIH-inner_corner-128-0.9-0.5-0.4-trial_3-29.png}}
    \label{fig:2DBPCAIH sample class evolution high rho high delta}
    \caption{Sample classroom evolutions for the inner corner configuration with $L=128$ at different times $t$ for positional learning rate $\rho_0=0.9$, $\delta\lambda = 0.4$.}
 \end{figure}

\subsection{$m$ vs $\rho$}\label{subsec:BPCAIH m vs rho}

\subsection{$t_{max}$ vs $\rho_0$}\label{subsec:BPCAIH t vs rho}
Figure \ref{fig:2DBPCAIH t-rho plot} shows the time to learn $t_{max}$ is directly affected by the level of heterogeneity $\delta\lambda$. 
This means that the heterogeneity $\delta\lambda$ has a significant effect on the time to learn $t_{max}$, with lower values of $\delta\lambda$ leading to lower time to learn $t_{max}$ for all values of $\rho_0$. 
We also notice that the traditional model is more affected by classroom heterogeneity comapred to PI models.

When aggregating the different results for each class size $L$, as shown in Figure \ref{fig:2DBPCAIH t-rho ribbon plot}, it shows affirms our previous findings that PI models are better for smaller clases and lower values of $\rho_0$.
This stays consistent when comparing the performance of a similar classroom size $L$ and heterogeneity $\delta\lambda$ using the traditional model.
However, the comparison is not as clear when comparing between different levels of heterogeneity $\delta\lambda$.
When $L\geq64$, the PI models can perform better than the traditional model depending on the heterogeneity $\delta\lambda$, even with the same positional learning factor $\rho_0$.

\begin{figure}[htbp!]
    \centering
    \subfigure[Outer corner $L=48$]{\label{fig:2DBPCAIH t-rho plot OC48}\includegraphics[width=0.45\textwidth]{figures/2D-BPCAIH-analysis/t-plots/t-outer_corner-48.png}}
    \subfigure[Outer corner $L=96$]{\label{fig:2DBPCAIH t-rho plot OC96}\includegraphics[width=0.45\textwidth]{figures/2D-BPCAIH-analysis/t-plots/t-outer_corner-96.png}}
    \subfigure[Inner corner $L=48$]{\label{fig:2DBPCAIH t-rho plot IC48}\includegraphics[width=0.45\textwidth]{figures/2D-BPCAIH-analysis/t-plots/t-inner_corner-48.png}}
    \subfigure[Inner corner $L=96$]{\label{fig:2DBPCAIH t-rho plot IC96}\includegraphics[width=0.45\textwidth]{figures/2D-BPCAIH-analysis/t-plots/t-inner_corner-96.png}}
    \subfigure[Traditional $L=48$]{\label{fig:2DBPCAIH t-rho plot T48}\includegraphics[width=0.45\textwidth]{figures/2D-BPCAIH-analysis/t-plots/t-traditional-48.png}}
    \subfigure[Traditional $L=96$]{\label{fig:2DBPCAIH t-rho plot T96}\includegraphics[width=0.45\textwidth]{figures/2D-BPCAIH-analysis/t-plots/t-traditional-96.png}}
    \caption{Time to learn $t_{max}$ as a function of positional learning factor $\rho_0$ for different representative classroom configurations and sizes. Lower time to learn $t_{max}$ indicates better performance.}
    \label{fig:2DBPCAIH t-rho plot}
 \end{figure}

 \begin{figure}[htbp!]
    \centering
    \subfigure[$L=32$]{\label{fig:2DBPCAIH t-rho ribbon plot 32}\includegraphics[width=0.45\textwidth]{figures/2D-BPCAIH-analysis/rho-t ribbon plots/rho-t-ribbon-32.png}}
    \subfigure[$L=48$]{\label{fig:2DBPCAIH t-rho ribbon plot 48}\includegraphics[width=0.45\textwidth]{figures/2D-BPCAIH-analysis/rho-t ribbon plots/rho-t-ribbon-48.png}}
    \subfigure[$L=64$]{\label{fig:2DBPCAIH t-rho ribbon plot 64}\includegraphics[width=0.45\textwidth]{figures/2D-BPCAIH-analysis/rho-t ribbon plots/rho-t-ribbon-64.png}}
    \subfigure[$L=96$]{\label{fig:2DBPCAIH t-rho ribbon plot 96}\includegraphics[width=0.45\textwidth]{figures/2D-BPCAIH-analysis/rho-t ribbon plots/rho-t-ribbon-96.png}}
    \subfigure[$L=128$]{\label{fig:2DBPCAIH t-rho ribbon plot 128}\includegraphics[width=0.45\textwidth]{figures/2D-BPCAIH-analysis/rho-t ribbon plots/rho-t-ribbon-128.png}}
    \label{fig:2DBPCAIH t-rho ribbon plot}
    \caption{Time to learn $t_{max}$ as a function of positional learning factor $\rho_0$ for the heteregenous model of all seating arrangements and classroom sizes $L$. Lower time to learn $t_{max}$ indicates better performance.}
 \end{figure}

\subsection{$m$ vs $\delta\lambda$}\label{subsec:BPCAIH m vs dl}

\subsection{$t$ vs $\delta\lambda$}\label{subsec:BPCAIH t vs dl}

\section{Anisotropic Interactions}
