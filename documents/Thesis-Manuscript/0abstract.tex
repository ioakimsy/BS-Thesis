% !TEX root =  main.tex
Peer instruction has recently become one of the popular means of classroom instruction in Physics Education. Such educational setup must involve both physical interaction with things and actually doing some procedural steps mentally or physically. In this study, we investigate the effects of different seating arrangements on the students' learning efficiency in peer instruction by modeling the transfer of knowledge within the class as a probabilistic cellular automata model.  We compared the efficiency of learning between the traditional learning model and the peer instruction model. We found that in square classrooms with different lengths $L \in\lbrace32,48,64,96,128\rbrace$, the inner corner seating arrangement performed the best among the peer instructions setups in terms of both the time $t_{max}$ it takes for all the students to learn and the classroom’s learning rate $m$. This result is different from a previous study, where they found that the outer corner seating arrangement performed the best. The difference stems from the simplifications made in this model that may not reflect real world factors. Our model uses binary values in an isotropic system and does not consider memory or unlearning. However, despite these simplifications, we found that in smaller classrooms with slow learners, peer instruction is more efficient compared to the traditional learning model, just as previous studies have suggested.

Taken from SPP-2024 paper
