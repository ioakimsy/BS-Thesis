% !TEX root =  main.tex
Peer-to-peer instruction has recently become one of the popular means of classroom instruction in Physics Education. Such educational setup must involve not only physical interaction with things but also actually doing some procedural steps mentally or physically. In this study, we investigate the effects of different seating arrangements to the students' learning efficiency in the peer-to-peer mode of instruction by modelling the transfer of knowledge within the class.  We compared the efficiency of learning between the traditional learning model and peer-to-peer learning model. We found that in square classrooms with different lengths $L \in\lbrace32,48,64,96,128\rbrace$, the inner corner seating arrangement performed the best among the peer-to-peer learning setups in terms of both the time it takes to saturate the classroom with learned students ($t_{max}$) and the classroom’s learning rate ($m$). This result is different from a previous studies. The difference stems from the simplifications made in this model that may not reflect real world factors. Our model uses binary values instead of continuous values as a measure of students’ aptitude, does not consider memory or unlearning and directionality or orientation bias (non-isotropy). It also does not consider the similarity effect mentioned in related literature where peer discussion can enhance understanding even if none of the students knows the correct answer. However, despite these simplifications, we found that in smaller classrooms with slow learners, peer-to-peer learning is more efficient compared to the traditional learning model, just as previous studies has suggested.

Taken from SPP-2024 paper
