% !TEX root =  main.tex
Peer instruction (PI) has recently become one of the popular means of classroom instruction in Physics Education. 
Such instruction method is vastly different from how classes are usually handled where the instructor conducts a lecture for the entire duration of the class.
In this study, we investigate the effects of different factors such as seating arrangements, size, learning rate, and heterogeneity on the classes' overall learning efficiency in peer instruction by modeling the transfer of knowledge within the class as a probabilistic cellular automata model. 
We compared the efficiency of learning between the traditional learning model and the PI model. 
We found that higher class sizes sway the advantage towards traditional instruction, while an increase in learning rate heterogeneity favors PI.
Additionally, an increase in the students' effective learning rate benefit both traditional instruction and PI but in different ways.
Classes under traditional instruction were found to have two stages of learning when heterogeneity was introduced: a fast initial stage and a slow final stage.
On the other hand, learning trends in PI were generally unaffected by heterogeneity, having similar effects with the other factors we considered.
Among the seating arrangements (SAs) we considered for PI, the inner corner seating arrangement performed the best in terms of both the time it takes for all the students to learn and the classroom’s learning rate.
This result differs from previous studies where they found that the outer corner SA performed the best.
The difference stems from the simplifications made in this model that may not reflect real world factors. Our model uses binary values in an isotropic system and does not consider the effect of aptitude similarity that have been described in previous researches. 
However, despite these simplifications, our other findings agree with previous studies and existing practices that PI performs similarly or better than traditional instruction and that a mix of traditional instruction and PI would be the optimal method of instruction.

