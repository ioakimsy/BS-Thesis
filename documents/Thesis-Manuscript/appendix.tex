% !TEX root =  main.tex
\chapter{Appendix}

\section{Derivation of Equation \ref{eq:BPCA PI learning probability}} \label{sec: BPCA PI learning probability derivation}
\begin{subequations}
    For any event $e$, the desired outcome occurs with probability $p$ or not with probability $q$ where

    \begin{equation}
        \label{eq:A1}
        p + q = 1.
    \end{equation}

    \noindent For $n$ events, each being event $e$:
    \begin{equation}
        \label{eq:A2}
        \prod_{\forall e}(p_e + q_e) = 1.
    \end{equation}

    \noindent Expanding equation \ref{eq:A2}, we get
    \begin{equation}
        \label{eq:A3}
        \prod_{\forall e}{p_e}+ \ldots + \prod_{\forall e}{q_e} = 1.
    \end{equation}

    \noindent where the sum of the first $n-1$ terms is the probability of the desired outcome occurring at least once over $n$ events and the last term is the of the probability of the desired outcome not occurring at all. Thus, we can rewrite the probability of the desired outcome occurring at least once as

    \begin{equation}
        \label{eq:A4}
        P = 1 - \prod_{\forall e}{q_e}.
    \end{equation}

    \noindent substituting equation \ref{eq:A1} into equation \ref{eq:A4}, we get

    \begin{equation}
        \label{eq:A5}
        P = 1 - \prod_{\forall e}{(1 - p_e)}.
    \end{equation}
\end{subequations}

\section{Julia packages used}
This study used the following \verb$Julia$ packages and their hash codes:

% \begin{enumerate}
%     \item \verb$Alert.jl$ % can't find citation
%     \item \verb+CSV.jl+ % can't find citation
%     \item \verb$CairoMakie.jl$ 
%     \item \verb$ColorSchemes.jl$ % can't find citation
%     \item \verb$DataFrames.jl$ 
%     \item \verb$GLMakie.jl$ \cite{MakieJL}
%     \item \verb$LaTeXStrings.jl$ % can't find citation
%     \item \verb$LsqFit.jl$ % can't find citation
%     \item \verb$Measurements.jl$ \cite{Measurements.jl-2016}
%     \item \verb$Plots.jl$ \cite{PlotsJL}
%     \item \verb$ProgressMeter.jl$ % can't find citation
% \end{enumerate}

\begin{enumerate}
    \item \verb+Alert = "28312eec-4d86-447d-83ad-bc2b262de792"+
    \item \verb+CSV = "336ed68f-0bac-5ca0-87d4-7b16caf5d00b"+
    \item \verb+CairoMakie = "13f3f980-e62b-5c42-98c6-ff1f3baf88f0"+ \cite{MakieJL}
    \item \verb+ColorSchemes = "35d6a980-a343-548e-a6ea-1d62b119f2f4"+
    \item \verb+DataFrames = "a93c6f00-e57d-5684-b7b6-d8193f3e46c0"+ \cite{DataFramesJL}
    \item \verb+GLMakie = "e9467ef8-e4e7-5192-8a1a-b1aee30e663a"+ \cite{MakieJL}
    \item \verb+LaTeXStrings = "b964fa9f-0449-5b57-a5c2-d3ea65f4040f"+
    \item \verb+LsqFit = "2fda8390-95c7-5789-9bda-21331edee243"+
    \item \verb+Measurements = "eff96d63-e80a-5855-80a2-b1b0885c5ab7"+ \cite{Measurements.jl-2016}
    \item \verb+Plots = "91a5bcdd-55d7-5caf-9e0b-520d859cae80"+ \cite{PlotsJL}
    \item \verb+ProgressMeter = "92933f4c-e287-5a05-a399-4b506db050ca"+
\end{enumerate}

\section{Codes as appendix}

idk how to do this with julia code na may unicode characters, will insert a github link or something before submission

% \hspace{\parindent}Include your codes when necessary to your
% thesis/dissertation. To do this, you may use \verb+verbatim+
% environment as follows. \textbf{WARNING:} All verbatim and
% verbatiminput environments should always be treated as a separate
% paragraph.  When included in a text paragraph, it sometimes happen
% to reduce the 1.5 spacing to the usual single-spaced text.

% {\small
% \begin{verbatim}
%  #include <iostream>
%  using std::cout;
%  using std::endl;

%  int main( void )
%  {
%     cout << "Hello world!" << endl;
%     return 0;
%  }
% \end{verbatim}
% }

% The \verb+{\small }+ bracketed region is used to lower the font
% size of the entire verbatim text.  This will save you much space
% and give a more aesthetical look in your manuscript.

% On the other hand, when very long codes are wished to be included
% automatically without the tedious cut and paste procedure, you may
% include them using the \verb+\verbatiminput+ command as follows.
% You may want to include a short description of the code of course.

% {\small
%  \verbatiminput{"codes/newC.cpp"}
% }

% This time, you may just include your recent codes by just
% copy-paste-ing the codes (as long they are clean!) into the
% directory \verb$codes/$ in the directory where this file is saved.
